% Hier ist abschließend ein prägnanter Überblick über die Problemstellung, die Ergebnisse und deren Interpretation zu geben. Jede Arbeit hat bestimmte Beschränkungen und ungelöste Fragen. An dieser Stelle sollten die wesentlichen Grenzen der Untersuchung aufgezeigt werden. Aufgrund dieser Erkenntnisse sollte der/die VerfasserIn auch Hinweise auf zukünftige Arbeiten und mögliche Verbesserungen geben. Die Zusammenfassung sollte 2 A4-Seiten nicht überschreiten.
\chapter{Zusammenfassung}
	%Die Problemstellung einen funktionierenden Arm zu entwerfen, der einen Billardstoß ausführen kann, wurde lange und durchdacht versucht zu lösen.
	%Nun liegt eine Konstruktion vor, welche in Rahmen der Lehrveranstaltung "`Gerätekonstruktion"' so gut wie möglich ausgearbeitet wurde.\par \medskip
	%Rückblickend ist zu sagen, dass bei der Konstruktion sehr viel Wert auf Funktionstüchtigkeit in verwirklichter Form gelegt wurde.
	%Deshalb können wir guten Gewissens behaupten, dass das Ziel der Aufgaben- und Problemstellung erreicht wurde.\\
	%Inwieweit das Gerät in der Praxis funktioniert, ist noch eine offene Fragestellung.
	%Dieser Frage könnte im Rahmen einer Projektarbeit beispielsweise nachgegangen werden.

	Bei dem konstruierten Arm zur Abbildung eines Billardstoßes konnten alle Festforderungen, Mindestforderungen und Wunschanforderungen eingehalten werden.
	Aufgrund der modularen Bauweise können alle Bauteile in kurzer Zeit aufgebaut und auch in entsprechenden Transportboxen verstaut werden.
	Des Weiteren liegt das Gesamtgewicht der Baugruppe unter dem Maximalgewicht von \SI{35}{\kilo\gram}.
	Durch die Konstruktion der Arme aus zwei Rohren und Fahrradklemmen kann außerdem die Länge der Arme zwischen \SI{25}{\cm} und \SI{30}{\cm} variiert werden.
	Die Stoßkraft und Stoßgeschwindigkeit kann über die pneumatischen Muskeln bzw. deren pneumatischer Steuerung, den individuellen Bedürfnissen angepasst werden.
	Auch die Aufnahme unterschiedlicher Queues wurde bei der Konstruktion berücksichtigt, indem die Klemme mit einer entsprechend weiten Öffnung gewählt wurde.
	Einsetzbar an diversen Orten und Tischen ist die Baugruppe durch eine Versorgung des Systems mit \SI{230}{\volt}, welche in jeder Billardhalle zur Verfügung stehen sollte. 
	Durch die Einstellgenauigkeit des Schultergelenks von einer Auflösungsgenauigkeit von \SI{0,06}{\degree} kann der Winkel immer an die Haltung des jeweiligen Spielers angepasst werden.
	Aufgrund des Aufbaus aus größtenteils Kaufteilen und 3D-gedruckten Teilen kann die Konstruktion auch kostenorientiert und zeitnah repariert werden, sollte es je zu defekten an der Baugruppe kommen.
	Dennoch gibt es einige Stellen der Konstruktion, die in einer nächsten Version optimiert werden könnten.
	So ist zum Beispiel die Befestigung am Dummy nur möglich, wenn ein Teil der Effektor-Baugruppe demontiert wird.
	Dies führt zu unnötig langen Rüstzeiten beim Einsatz des Systems.
	Außerdem muss sich zeigen, ob alle 3D-gedruckten Teile einer dauerhaften Belastung standhalten.
	Weiter sind die Fahrradklemmen, die einen großen Teil der Konstruktion zusammenhalten, sehr grob und bringen ein großes Gewicht mit sich.
	Sie könnten durch flachere und filigranere Schellen ersetzt werden. \\
	Auch sind noch einige Punkte offen.
	Der komplette Aufbau des Steuerungssystems und die fachgerechte Verlegung der Pneumatikleitungen sind noch nicht erfolgt. 
	Außerdem müsste das Steuerungssystem noch in einem Gehäuse verstaut werden sowie ein Human Interface zur Bedienung des Gerätes erstellt werden. \\
	Generell aber sind wir von unserer Konstruktion überzeugt und sind der Meinung, dass sie bei der korrekten Ausführung und Überwachung von Stößen beim Billard spielen helfen kann. 
	Sie kann kostengünstig gefertigt werden und ist so für viele Vereine attraktiv.
	Auch die schnelle Reparaturmöglichkeit sehen wir als großen Vorteil an.
