% Hier ist abschließend ein prägnanter Überblick über die Problemstellung, die Ergebnisse und deren Interpretation zu geben. Jede Arbeit hat bestimmte Beschränkungen und ungelöste Fragen. An dieser Stelle sollten die wesentlichen Grenzen der Untersuchung aufgezeigt werden. Aufgrund dieser Erkenntnisse sollte der/die VerfasserIn auch Hinweise auf zukünftige Arbeiten und mögliche Verbesserungen geben. Die Zusammenfassung sollte 2 A4-Seiten nicht überschreiten.
\chapter{Zusammenfassung}
	Die Problemstellung einen funktionierenden Arm zu entwerfen, der einen Billardstoß ausführen kann, wurde lange und durchdacht versucht zu lösen.
	Nun liegt eine Konstruktion vor, welche in Rahmen der Lehrveranstaltung "`Gerätekonstruktion"' so gut wie möglich ausgearbeitet wurde.\par \medskip
	Rückblickend ist zu sagen, dass bei der Konstruktion sehr viel Wert auf Funktionstüchtigkeit in verwirklichter Form gelegt wurde.
	Deshalb können wir guten Gewissens behaupten, dass das Ziel der Aufgaben- und Problemstellung erreicht wurde.\\
	Inwieweit das Gerät in der Praxis funktioniert, ist noch eine offene Fragestellung.
	Dieser Frage könnte im Rahmen einer Projektarbeit beispielsweise nachgegangen werden.