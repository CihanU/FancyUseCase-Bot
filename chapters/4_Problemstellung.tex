% Ausgehend von den in der Literaturbesprechung identifizierten Defiziten im Stand der Technik wird in diesem Teil erklärt, wie und warum ein bestimmtes Problem ausgewählt wurde, wer es als problematisch empfindet und welche Bedeutung die Problematik hat. Es sollte dargestellt werden, welchen Beitrag die Bearbeitung des Themas einerseits zur Forschung (Theorie, Modell, Methoden, Fakten) leistet und andererseits welche praktische Relevanz mit dem Thema verbunden ist. Da oft nicht alle Aspekte eines Problems im Rahmen einer wissenschaftlichen Arbeit behandelt werden können, wird die Problemstellung durch eine konkretisierte Zielsetzung eingegrenzt. Sie ergibt sich aus dem Forschungsinteresse sowie Erwägungen zur Machbarkeit und zum vertretbaren Aufwand. Kurz und klar sollte aufgeführt werden, was erreicht werden soll, welche Ergebnisse zu welchem Verwendungszweck angestrebt werden, und welche Art von Schlussfolgerungen in Bezug auf das Gesamtproblem daraus möglich werden sollen.
\chapter{Aufgabenstellung}\label{chap:Aufgabenstellung}

	Zu konstruieren ist ein System, das einen Billard-/Snookerstoß nachbilden kann.
	Es soll die Stoßparameter der von durchschnittlichen Spielern durchgeführten Stöße möglichst exakt abbilden während die Ergonomie des menschlichen Bewegungsapparates wo nötig nachgeahmt wird.
	Weiter ist das System für den mobilen Einsatz an realen und beliebigen Snookertischen vorgesehen. Im Rahmen einer gewünschten Einsetzbarkeit in Gegenwart von ungeschultem Personal sind Aufhängungs- bzw. Standsicherheit, sowie ein sicherer Betrieb zu gewährleisten.
	Zum Transport muss das Gerät zerlegbar und in maximal drei Transportkisten von jeweils höchstens \SI{42}{L} Fassungsvolumen verstaubar sein.

	\section{Anforderungskatalog}
		Aus den oben formulierten Anforderungen soll im Weiteren eine konkretisierte Auflistung der Fest- Mindest- und Wunschanforderungen erstellt werden.
		Darüber hinaus sollen implizite Anforderungen in verbalisierter Form aufgeführt sein.\par\medskip

		\textbf{Festanforderungen:}
		\begin{itemize}
			\item Einsetzbarkeit an beliebigen Tischen und Orten
			\item Autarke Funktion
			\item Einstellbarkeit der Parameter
			\begin{itemize}
				\item Armlängen
				\item Stoßwinkel
				\item Stoßkraft
				\item Stoßgeschwindigkeit
			\end{itemize}
			\item Aufhäng- bzw. Standsicherheit
			\item Aufnahme verschiedener Queues
			\item Beachtung der Sicherheitsanforderungen für Bedienpersonal
		\end{itemize}

		\textbf{Mindestanforderungen:}
		\begin{itemize}
			\item Impulsübertrag von Queue auf Snookerkugel von
			\begin{itemize}
				\item Durchmessern \SIrange{38}{68}{\milli\metre} und
				\item Massen \SIrange{45}{204}{\gram}
			\end{itemize}
			\item Transportierbarkeit in maximal drei Boxen mit Volumina von jeweils maximal \SI{42}{L}
			\item Maximales Gesamtgewicht von \SI{35}{\kilo\gram}
			\item Maximale Aufbauzeit durch eine Person von \SI{60}{\minute}
			% \item Maximalkosten für fünf Einheiten von 5.000,- EUR
		\end{itemize}

		\textbf{Wunschanforderungen:}
		\begin{itemize}
			\item Nachbildung der menschlichen Motorik von Schulter bis Handgelenk
			\item Nachbildung der Stoßparameter eines durchschnittlichen Snookerspielers
		\end{itemize}