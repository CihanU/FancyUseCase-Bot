% Dieses Kapitel bildet zusammen mit dem folgenden Kapitel Diskussion den Hauptteil der Arbeit. In übersichtlicher Gliederung und sinnvoller Reihenfolge wird dargestellt, was mittels der eingesetzten Methodik im Hinblick auf die Zielsetzung herausgefunden werden konnte. Die Strukturierung des Kapitels orientiert sich an der Theorie und Methodik. Mit Hilfe von Grafiken und Diagrammen (siehe 2.8.4) werden die Ergebnisse wertfrei dargestellt. Eine Interpretation erfolgt an dieser Stelle noch nicht.
\chapter{Ergebnisse}
	Das auf Basis der vorangegangenen Überlegungen entwickelte Gerät soll in den folgenden Kapiteln anhand technischer Zeichnungen näher erläutert werden.
	Eine vollständige technische Dokumentation ist in \cref{sec:zeichnungen} zu finden.
	\section{Konzeption und Entwurf}
		Nun werden die zur Konstruktion konzipierten Bauteile allesamt mit Zeichnungen präsentiert und kommentiert.
		Dabei wird unter anderem sowohl auf das Prozedere der Konstruktion als auch auf die aufgetauchten Probleme und die Vorgehensweise zur Lösung jener eingegangen.\\
		Die technischen Zeichnungen zu den Bauteilen sind jeweils in \cref{zeichnungen} angehängt.
		Die Übersichtszeichnungen sind meist explosionsartige Darstellungsformen und enthalten Nummerierungen, die dann jeweils in der darauffolgenden Stückliste manifestiert sind.\\
		Die Zeichnungen zu dem Gesamtbauteil befinden sich in \cref{bauteilzeichnungen}, die des Schultergelenks in \cref{zeichnungen-schulter}, die Einzelteile des Ober-/Unterarms in \Cref{zeichnungen-arme} und die des Ellbogengelenks in \cref{zeichnungen-ellbogen}.
		Zu dem befinden sich in \cref{stuckliste-pneumatik} weitere sonstige Stücklisten und außerdem der Schaltplan für die Pneumatik.
		
	\section{Gesamtbauteil}
		Wie vorhin bereits erwähnt, sind die Zeichnungen des Gesamtbauteils im \cref{bauteilzeichnungen} angehängt.
		Diese zeigen das fertige Konstrukt als Ganzes und auseinander geschraubt.
		Die Maße und Dimensionierungen der Einzelteile sind den Zeichnungen der jeweilig zugehörigen Unterkapiteln zu entnehmen.\\
		Allgemein ist zu sagen, dass die Bauteile so konstruiert werden mussten, dass sie einander angepasst sind.
		Dabei spielen die Verbindungen zwischen den einzelnen Gliedern eine entscheidende Rolle.
		Das Gesamtbauteil besteht aus Schulter, Oberarm, Muskeln (Bizeps/Trizeps), Ellbogen, Unterarm und Hand.
	
	\section{Schulter}
		Die Schulter hat die Aufgabe sowohl den gesamten Arm am Dummy zu befestigen als auch die Bewegung des Arms zu initialisieren.
		Für diese Aufgabe wurde das Zykloidalgetriebe gewählt und konstruiert.
		Das Zykloidalgetriebe ist ein mechanisches Getriebe, das die Umdrehungen reduziert, indem es die exzentrische Bewegung des Zykloidenrads einstellt, welches die Ausgangswelle antreibt. Es besitzt ein hohes Übersetzungsverhältnis und ist somit gut geeignet für unsere Anwendung.
		Das Schultergelenk ist ebenfalls in \cref{bauteilzeichnungen} und weitere Einzelheiten in \cref{zeichnungen-schulter} zu betrachten.
	
	\section{Ellbogen}
		Der Ellbogen fungiert als Verbindung von Ober- und Unterarm und als Halterung- bzw. Stützpunkt des Trizeps.
		Er ist unter \cref{bauteilzeichnungen} genauer anzusehen.
		Das Ellbogengelenk ist mit gekauften und gedruckten Teilen erstellt.
		Dabei werden alle Teile, die käuflich erworben werden können, auch zugekauft.
		Die selbst konstruierten Einzelteile wie das Rohrteil ist in \cref{zeichnungen-ellbogen} zu sehen.
	
	\section{Ober-/Unterarm}
		Die Konstruktion des Ober- und Unterarms wurde zunächst an den Prinzipskizzen orientiert erstellt.
		Dazu wurden entsprechende Bauteile in CAD gezeichnet und als Baugruppe zusammengefügt.
		Allerdings zeigte sich nach ersten Berechnungen zu den künstlichen Muskeln, dass diese Konstruktion deutlich zu schwer war, um sie mit der Kraft der künstlichen Muskeln zu bewegen.
		Nach weiteren Recherchen wurde außerdem klar, dass die Bauteile einen extrem hohen Preis in der Fertigung auswiesen.
		Aus diesen Gründen musste nach kurzer Zeit die Konstruktion komplett überarbeitet werden.
		Dabei wurde besonders darauf geachtet, dass sich sowohl das Gewicht möglichst gering hält als auch die Kosten durch die Verwendung von Kaufteilen reduzieren.
		Dies führte dazu, dass die Konstruktion nahezu komplett aus Aluminiumrohren und 3D-gedruckten Teilen konstruiert wurde.
		Dabei liegt auch der Vorteil, dass nicht alle Bauteile eigenständig konstruiert werden müssen.
		So lassen sich zum Beispiel die Klemmen, die die Rohrteile miteinander verbinden, als gewöhnliche Fahrradklemmen ausführen.
		Auch die Anschlüsse der künstlichen Muskeln können als 3D-gedruckte Teile ausgearbeitet werden und mit einer einfachen Rohrschelle an die Arme befestigt werden.
		Spezielle Teile wie die Einsätze der Rohre wurden eigenständig konstruiert.\\
		Der Ober- und Unterarm als Ganzes ist in \cref{bauteilzeichnungen} zu sehen und die Einzelteile wie die Innen- und Außenrohre sind in \cref{zeichnungen-arme} zu betrachten.
	
	\section{Hand}
		Das Handgelenk sowie die Hand (abgebildet in \cref{bauteilzeichnungen}) bestehen im Wesentlichen aus gekauften Teilen.
		Dabei wird das Handgelenk so ausgelegt, dass es sowohl mit einer Klemme mit einer Kugelform als Anschluss ausgestattet werden kann als auch mit einer Schraube der Größe M5 befestigt werden kann.
		Dabei ist lediglich zu beachten, dass die Kugelform mit einem Durchmesser von 28 mm ausgeführt ist.
		Die Klemme bzw. Hand sollte des Weiteren breit genug ausgearbeitet sein, um dem Queue auch eine entsprechende seitliche und vertikale Stabilisation zu ermöglichen.
	
	\section{Künstliche Muskeln}
		Die künstlichen Muskeln bilden hauptsächlich der Bizeps und der Trizeps, welche ebenso in \cref{bauteilzeichnungen} zu sehen sind.
		Beide sind jeweils mit einem Ende am Oberarm und mit dem anderen Ende am Unterarm verbunden.
		Die Muskeln funktionieren durch pneumatischen Antrieb.
		Diese Pneumatik verlangt eine Verschaltung, wie sie in \cref{stuckliste-pneumatik} schematisch dargestellt ist.