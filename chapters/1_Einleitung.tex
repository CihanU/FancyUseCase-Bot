% Die Einleitung umfasst eine Hinführung zum Thema und sollte die Relevanz der Aufgabenstellung in ihrem wissenschaftlichen Kontext herausstellen. Darüber hinaus sollte an dieser Stelle Bezug auf die Problemstellung genommen und dem Leser ein kurzer Ausblick auf die Arbeit vermittelt werden (in der Form: „Was erwartet den Leser?“). Der Umfang soll 3 Seiten nicht übersteigen.
\chapter{Einleitung}
	
	Billard ist ein Spiel, in dem Kugeln durch Stoßen einer anderen, weißen Kugel in die Taschen an den Ecken und den langen Seiten des Billardtischs versenkt werden. Dabei führt der Spieler mit einem Arm durch Führen des sogenannten Queues einen Stoß durch, während es von der Hand des anderen Arms gestützt wird.\\
	Profispieler streben durch Sammeln von Erfahrung und Expertise nach einer perfekten Haltung. Dabei spielt der Arm eine ganz wichtige Rolle, da er fast für den ganzen Erfolg des Stoßes verantwortlich ist. Genau deshalb ist es für einen Spieler auch wichtig seine Haltung analysieren zu können und Verbesserungsmöglichkeiten aufgezeigt zu bekommen.\\
	Dies soll durch einen den Spieler unterstützenden Mechanismus geschehen. Es ist ein System gegeben, welches den Stoß eines Spielers aufzeichnen und zur Nachahmung weitervermittelt werden kann. Durch die Aufzeichnung des Verhaltens ist der Spieler in der Lage, sinnvolle Rückmeldung zu seiner Haltung, Durchführung und Sonstiges zu erhalten und diese zu verbessern. Um dies zu erreichen, muss eine dem Menschen ähnliche Struktur des Arms realisiert und zur Anwendung bereitgestellt werden.\\
	Um diesen Anforderungen gerecht zu werden, wird in diesem Bericht eine Lösungsmethode für die aufkommende Problemstellung ausgearbeitet. Dabei wird darauf geachtet, dass die Anatomie und Ergonomie eines echten Menschen beim Stoß einer Billardkugel mit einem Queue berücksichtigt wird.\\
	Idealerweise wird eine Konstruktion erhalten, welche alle Fähigkeiten und Eigenschaften eines Menschen beinhaltet und diese je nach Bedarf angepasst werden können.
