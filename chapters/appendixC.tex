%LTeX: language=DE
\chapter{Aufbauanleitung}
Zunächst werden die Einzelteile zusammengebaut. 
Zur Veranschaulichung dienen dabei die Explosionszeichnungen. \par \medskip
Der Aufbau folgt dabei der Reihenfolge der Baugruppennummern. 
Zunächst werden allerdings die zwei Unterbaugruppen Ellbogengelenk und Sehne beschrieben.
Diese sollten auch als Erstes zusammen gebaut werden.
Das Ellbogengelenkt besteht im Wesetlichen aus den zwei Rohreinsätzen für die Arme sowie einem Gelenk.
Als Erstes werden die vier Einpressgewindehülsen in die Rohrteile eingesetzt. 
Dazu wird ein Lötkolben benötigt und die Hülsen in die entsprechenden Bohrungen eingeschmolzen. 
Nun kann das Schanier mit den langen Schrauben auf der anderen Seite der Rohrteile angebracht und über die Einpresshülsen verschraubt werden. 
In die große Bohrung wird mithilfe von Allzweckkleber die Gewindestange mit der Bohrung nach oben in das Rohrteil eingeklebt. 
An der Gewindestange werden mit vier Unterlegscheiben und einer Schraube der Muskelanschluss für den Trizeps angeschraubt. 
Diese Stelle muss aufgrund ihrer zwingendnotwendigen Beweglichkeit mit Fett geschmiert werden, außerdem darf die Schraube unter keinen Umständen fest angezogen werden. 
Sie wird lediglich beigelengt und über die Reibung der selbstsichernden Mutter gehalten. \par
Nun erfolgt die Montage der Sehne. 
Diese besteht aus den zwei Muskelanschlüssen. Einmal statisch, diese wird am Rohr der Arme fixiert, und einmal drehbar, diese ist letztendlich mit dem künstlichen Muskel verbunden. 
Auch um an dieser hochbelasteten Stelle die Reibung gering zuhalten wird zum einen an jeder Verbindungsstelle eine Unterlegscheibe eingesetzt un dzusätzlich werden diese Stellen ebenfalls gefettet.
Hier gilt aufgrund der benötigten Beweglichkeit ebenfalls, dass die Schrauben nicht fest angezogen werden dürfen, sondern lediglich beigelegt werden. 
Diese Baugruppe wird insgesamt dreimal aufgebaut. \par
Im Anschluss erfolgt der Aufbau des Zykloidalgetriebes.
Dabei sind die Positionen der Lager und die Reihenfolge der Explosionszeichnung zu entnehmen.
Über das M10-Gewinde des Getriebes wird dieses an dem gestellten Dummy befestigt. \par
Der Oberarm wird aus den beiden Rohrteilen zusammengeschoben und mit den Fahrradklemmen miteinander verbunden.
Des Weiteren werden die bereits zusammengebauten Sehnen, bestehend aus zwei Muskelanschluss drehbar und statisch, mit einer Rohrschelle an dem größeren Rohr befestigt. 
Weiter wird im Außenrohr das Ellbogenlenk mit der entsprechenden Schraube im Rohr fixiert. 
Dabei muss das Gelenk schon zusammengebaut sein. 
Außerdem muss das Teil ohne die Gewindestange in das Oberarmrohr eingebaut werden.
Sobald die Baugruppe zusammengebaut wurde, kann sie mit einer weiteren Fahrradklemme an dem Schultergelenk, also dem Zykloidalgetriebe, befestigt werden. 
Nun folgt die Montage des Unterarms.
Dieser wird ähnlich wie der Oberarm aus zwei Rohren zusammengesetzt und mit einer Fahrradklemme verbunden. 
In das dünnere der beiden Rohre wird der Queue-Halter mit der entsprechenden Schraube befestigt. In die Queue-Halterung wird mit der Überwurfmutter das Queue-Gelenk eingesetzt. 
Dabei muss je nachdem, wie beweglich der Queue geführt werden soll, die Mutter mit Schraubensicherungen der mittleren Stufe, die Überwurfmutter bei der gewünschten Festigkeit verklebt werden. 
An dem Queue-Gelenk wird die Queue-Klemme mit einer Schraube der Größe M5 befestigt. 
Außerdem wird an dem dünneren Rohr ebenfalls eine Sehne mit der Schelle befestigt. 
Zuletzt werden die pneumatischen Muskeln in die entsprechenden Anschlüsse der Sehne eingeschraubt und so die Baugruppe komplettiert. 