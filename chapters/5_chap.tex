\chapter{Dimensionierung}

\begin{equation}
M = b \cdot F = I \cdot \alpha
\end{equation}

mit dem Trägheitsmoment \( I = m \cdot s_m^2 \), der Tangentialbeschleunigung \( a_T = \frac{v^2}{\Delta x} \) und der
Winkelbeschleunigung \( \alpha = \frac{a_T}{B} \) \par\medskip

\begin{align*}
I \cdot \frac{a_T}{B} &= m \cdot s_m^2 \cdot \frac{a_T}{B} \\
b \cdot F &= m \cdot s_m^2 \cdot \frac{a_T}{B} \\
F_x &=\frac{m \cdot {s_m}^2 \cdot a_T}{B \cdot b \cdot cos(\beta)}
\end{align*}

hier sind

\( m := \) Masse Unterarm \\
\( s_m := \) Massenschwerpunkt entlang $ B $ \\
\( v := \) Soll-Endgeschwindigkeit \\
\( \Delta x := \) Ausholstrecke vor dem Stoß \\
\( B := \) Strecke Drehachse - Ende des Unterarms \\
\( b:= \) Strecke Drehachse - Angriffspunkt PAM \\
\( \beta := \) Winkel PAM  Oberarm \\
\( F:= \) Erforderliche Kraft des PAM